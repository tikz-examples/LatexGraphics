% Drawing Neural Networks in TikZ: Short Guide
% latexdraw.com
% 21/02/2021 at 22:20

\documentclass[border = 0.2cm]{standalone}

% Required package
\usepackage{tikz}

\begin {document}

% Input layer neurons'number
\newcommand{\inputnum}{2} 
% Hidden layer neurons'number
\newcommand{\hiddennum}{5}  
% Output layer neurons'number
\newcommand{\outputnum}{2} 

\begin{tikzpicture}

% Input Layer
\foreach \i in {1,...,\inputnum}
{
	\node[circle, 
		minimum size = 6mm,
		fill=orange!30] (Input-\i) at (0,-\i) {};
}

% Hidden Layer
\foreach \i in {1,...,\hiddennum}
{
	\node[circle, 
		minimum size = 6mm,
		fill=teal!50,
		yshift=(\hiddennum-\inputnum)*5 mm
	] (Hidden-\i) at (2.5,-\i) {};
}

% Output Layer
\foreach \i in {1,...,\outputnum}
{
	\node[circle, 
		minimum size = 6mm,
		fill=purple!50,
		yshift=(\outputnum-\inputnum)*5 mm
	] (Output-\i) at (5,-\i) {};
}

% Connect neurons In-Hidden
\foreach \i in {1,...,\inputnum}
{
	\foreach \j in {1,...,\hiddennum}
	{
		\draw[->, shorten >=1pt] (Input-\i) -- (Hidden-\j);	
	}
}

% Connect neurons Hidden-Out
\foreach \i in {1,...,\hiddennum}
{
	\foreach \j in {1,...,\outputnum}
	{
		\draw[->, shorten >=1pt] (Hidden-\i) -- (Output-\j);
	}
}

% Inputs
\foreach \i in {1,...,\inputnum}
{            
	\draw[<-, shorten >=1pt] (Input-\i) -- ++(-1,0)
		node[left]{$x_{\i}$};
}

% Outputs
\foreach \i in {1,...,\outputnum}
{            
	\draw[->, shorten >=1pt] (Output-\i) -- ++(1,0)
		node[right]{$y_{\i}$};
}


\end{tikzpicture}

\end{document}